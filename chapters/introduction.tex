\chapter{Introduction}

\begin{center}
\textit{In the name of God, the Most Compassionate, the Most Merciful}
\end{center}

\section{The Temporal Nature of This World}

No human being is immortal. People are born and they pass away. This world, which in its literal Arabic sense (\textit{dunya}) denotes the lowest tier of existence, is our temporary lodging, not our eternal abode. A day will come when the footprints of each of us will vanish from this earth, and only the remembrance of us and the traces of our deeds will remain.

From ancient times, humanity has broadly been seen as two groups: the believers and the others. One group has faith in the One Creator, and another denies Him; and of course both groups encompass many shades and degrees. Believers hold that the world is not without a Maker, and that God has completed the proof for all people through inner signs (\textit{fitrah}) and outer signs (\textit{ayat}), so that on the Day of Reckoning they will not say, `We did not know.'

Among the believers are those who hold that Muhammad ibn Abd Allah (peace and blessings be upon him and his Household) is the last of the prophets. He was born around 570 CE in Mecca and, at about the age of forty, was charged to proclaim his mission, calling people to worship God and to obey His command in all affairs. He continued this call until the end of his life and left behind two precious legacies: the Book of God (the Qur'an) and the \textit{Ahl al-Bayt} (peace be upon them).

The Qur'an is a book of guidance: the Word of God revealed to His final Prophet. It confirms the earlier prophets and scriptures and clarifies matters of dispute: it makes `no distinction between any of His messengers.' In it, many of the issues over which the People of the Book differed are explained.

\section{Why `He is the One who\ldots' -- \textit{Huwa alladhi}?}

In the Qur'an there are many verses that begin with the phrase `\textit{Huwa alladhi}' (`He is the One who\ldots'). This structure beautifully reminds us of the divine attributes and acts. `He is the One who\ldots' is a prelude that brings signs of God's oneness and lordship, knowledge and wisdom, power and mercy, and governance. This book gathers every verse that begins with this phrase, one verse per page, along with a clear and fluent translation. The aim is to make reflection on the attributes and acts of the Lord, through the Qur'an itself, more accessible.

\section{How This Idea Took Shape}

I was reading the Qur'an. With each reading, the divine verses left a fresh impression on my heart and mind. Sometimes a single word, and sometimes a recurring structure, would make me pause. Reflection on the Qur'an is a boundless treasury; each time you look, a new horizon opens. One such moment was when I noticed how many verses begin with `\textit{Huwa alladhi}'. The structure caught my attention: sentences that start with `He is the One who\ldots' and then set out a facet of the divine attributes, acts, or governance---as though God is introducing Himself to His servants directly, without embellishment, in a firm and lucid tongue. From there the spark for this book was lit. What began as a personal aid for my own reflection gradually took on the shape of a project; perhaps this collection can also help others contemplate the Lord's attributes.

\section{Structure of the Book}

In this book, all verses that begin with `\textit{Huwa alladhi}' are presented exactly in the order they appear in the \textit{Mushaf}, without thematic grouping, one after another. This is a deliberate choice, so that the reader encounters the natural sequence of these verses within the fabric of revelation and better perceives each verse in its Qur'anic context. Each page contains one complete verse in Arabic along with its translation.

\section{Intended Audience}

This collection is for anyone who, from the Qur'an's vantage point, seeks a clearer understanding of God's attributes and acts: for believers and seekers of truth, for those who reflect, and for teachers and students interested in conveying Islamic concepts.

\section{Notes on the Translation}

For the translations, I have used the work of the late Dr. Tahereh Saffarzadeh. Every effort has been made to reproduce her translations accurately and without alteration.

\section{Final Words}

This collection is an invitation to reflect on the attributes of the Lord. `He is the One who\ldots'---and after this brief opening lies a world of meaning. May the reading of this book soften hearts, illuminate minds, and deepen faith. I close with the supplication before reciting the Qur'an as narrated from Imam al-Sadiq (peace be upon him).

\subsection{Prayer before reciting the Qur'an (Arabic)}

\begin{center}
\begin{Arabic}
اللّٰهُمَّ إِنِّى أَشْهَدُ أَنَّ هٰذَا كِتابُكَ الْمُنْزَلُ مِنْ عِنْدِكَ عَلَىٰ رَسُولِكَ مُحَمَّدِ بْنِ عَبْدِاللّٰهِ صَلَّى اللّٰهُ عَلَيْهِ وَآلِهِ، وَكَلامُكَ النَّاطِقُ عَلَىٰ لِسانِ نَبِيِّكَ جَعَلْتَهُ هادِياً مِنْكَ إِلىٰ خَلْقِكَ، وَحَبْلاً مُتَّصِلاً فِيما بَيْنَكَ وَبَيْنَ عِبادِكَ .اللّٰهُمَّ إِنِّى نَشَرْتُ عَهْدَكَ وَكِتابَكَ، اللّٰهُمَّ فَاجْعَلْ نَظَرِى فِيهِ عِبادَةً، وَقِراءَتِى فِيهِ فِكْراً، وَفِكْرِى فِيهِ اعْتِباراً، وَاجْعَلْنِى مِمَّنْ اتَّعَظَ بِبَيانِ مَواعِظِكَ فِيهِ، وَاجْتَنَبَ مَعاصِيَكَ، وَلَا تَطْبَعْ عِنْدَ قِراءَتِى عَلَىٰ سَمْعِى، وَلَا تَجْعَلْ عَلَىٰ بَصَرِى غِشاوَةً، وَلَا تَجْعَلْ قِراءَتِى قِراءَةً لَاتَدَ بُّرَ فِيها، بَلِ اجْعَلْنِى أَتَدَبَّرُ آياتِهِ وَأَحْكامَهُ آخِذاً بِشَرايِعِ دِينِكَ، وَلَا تَجْعَلْ نَظَرِى فِيهِ غَفْلَةً، وَلَا قِراءَتِى هَذَراً، إِنَّكَ أَنْتَ الرَّؤُوفُ الرَّحِيمُ.
\end{Arabic}
\end{center}

\subsection{Prayer before reciting the Qur'an (English translation)}

O Allah, I bear witness that this is Your Book, sent down from You to Your Messenger, Muhammad son of Abd Allah, may Allah bless him and his Household; and it is Your Speech, articulated upon the tongue of Your Prophet. You have made it a guide from You to Your creation, and a cord extending between You and Your servants. O Allah, I have opened Your covenant and Your Book. So make my looking into it an act of worship, my recitation of it reflection, and my reflection upon it a means of taking heed. Make me among those who are admonished by the clarifications of Your exhortations in it and who avoid disobedience to You. Do not seal my hearing at the time of my recitation, do not place a veil upon my sight, and do not make my recitation one in which there is no contemplation; rather, make me ponder its verses and its rulings, adhering to the ordinances of Your religion. Do not make my looking into it inattentive, nor my recitation rambling. Indeed, You are the Most Kind, the Most Merciful.

\vspace{2em}

\begin{flushright}
\textit{With best wishes,}\\
Milad Amirzadeh
\end{flushright}

